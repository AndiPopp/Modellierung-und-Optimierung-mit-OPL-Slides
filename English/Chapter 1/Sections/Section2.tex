\sectionframe{Linear Optimization}
\subsection{Properties}
\begin{frame}\small
 \frametitle{Linear functions and constraints}
 \begin{block}{Linear functions}
  \vspace{-2\baselineskip}
  \begin{equation*}
    f(x_1, \ldots, x_N) = \sum_{n=1}^{N} c_n\cdot x_n
  \end{equation*}
 \end{block}
 \vspace{-\baselineskip}
 \begin{block}{Linear constraints}
  Let $f$ be a linear function:
  \begin{align*}
   &f(x_1, \ldots, x_N) = b\\
   &f(x_1, \ldots, x_N) \leq b\\
   &f(x_1, \ldots, x_N) \geq b\\
  \end{align*}
 \end{block}
 \vspace{-2\baselineskip}
 \begin{block}{Linear optimization model}
  objective function and constraints linear in the decision variables $\Longrightarrow$ linear optimization model 
 \end{block}
\end{frame}

\begin{frame}
 \frametitle{Properties of linear functions}
 \begin{description}
  \item[Proportionality] Each variable contributes a proportional value to the function.
  \item[Independence] The value each variable contributes to the function is independent of the manifestation of the other variables.
 \end{description}
\end{frame}


\begin{frame}
 \frametitle{Typical signs for non-linearity}
 \begin{itemize}
  \item variables have other exponents than~$1$
  \begin{itemize}
   \item other natural exponents, e.g.: $x^2$
   \item roots, e.g.: $\sqrt{x} = x^{\frac{1}{2}}$
   \item variables in the denominator, e.g.: $\frac{1}{x} = x^{-1}$
  \end{itemize}
  \item variables are multiplied with each other, e.g.: $x_1\cdot x_2$
  \item exponential functions, e.g.: $2^x$
  \item absolute values, e.g. $|x|$
 \end{itemize}

 \begin{block}{Anomaly: constants}
  While constants are non-linear by definition, they do not interfere with linear optimization models, since they can be eliminated easily.
 \end{block}
\end{frame}

\begin{frame}
 \frametitle{deterministic vs. stochastic optimization models}
 \begin{description}
  \item[deterministic optimization models:] all parameters and function values are always exactly known
  \item[stochastic optimization models:] parameters and function values succumb to random deviations
 \end{description}
 
 Linear optimization models are deterministic in general.
\end{frame}

\begin{frame}
 \frametitle{continuous vs. integer optimization models}
 \begin{description}
  \item[continuous optimization models:] the values of the decision variables are continuous (real) values
  \item[integer optimization models:] the values of the decision variables can only be integer
 \end{description}
 
 \begin{block}{Types of linear optimization models by possible values for decision variables:}
  \begin{itemize}\footnotesize
   \item continuous decision variables $\Longrightarrow$ (continuous) linear optimization model
   \item integer decision variables $\Longrightarrow$ integer linear optimization model
   \item continuous and integer decision variables $\Longrightarrow$ mixed integer linear optimization model
  \end{itemize}
 \end{block}
\end{frame}

\subsection{Solution of linear problems}

\begin{frame}
 \frametitle{Solution structure of linear optimization models}
 \begin{itemize}
  \item there is exactly one optimal solution
  \item there is an unlimited number of optimal solutions
  \item there is no optimal solution
  \begin{itemize}
   \item the solution space is empty
   \item the solution space is unbound and the objective function approaches infinity
  \end{itemize}
 \end{itemize}
\end{frame}


\begin{frame}
 \frametitle{Well known solution methods for linear optimization models}
 \begin{block}{Solution methods for (continuous) linear optimization models}
  \begin{itemize}
    \item Dantzig's simplex method
    \item Karmarkar's inner point method
  \end{itemize}
 \end{block}
 \begin{block}{Solution methods for (mixed) integer linear optimization models}
  \begin{itemize}
    \item Branch and bound method
    \item Cutting planes methods
  \end{itemize}
 \end{block}
\end{frame}




