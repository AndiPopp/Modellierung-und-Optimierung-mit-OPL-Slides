\sectionframe{OPL in programming sequences}
\begin{frame}
 \frametitle{Example: cyclic staffing problem}
 \begin{block}{Algorithmic solutions for the cyclic staffing problem}\footnotesize
  \begin{enumerate}
   \item Let $P_1$ be the LP relaxation of the cylic staffing problem. Solve $P_1$, if solution is integer end, otherwise go to 2.
   \item Add the constraint \[\sum_{s\in S}x_s = \left\lfloor\sum_{s\in S}x^*_s\right\rfloor\] to $P_1$ to get $P_2$. If $P_2$ has a solution end, otherwise go to 3.
   \item Add the constraint\[\sum_{s\in S}x_s = \left\lceil\sum_{s\in S}x^*_s\right\rceil\] to $P_1$ to get $P_3$. An optimal solution of $P_3$ is always an optimal solution for the cyclic staffing problem.
  \end{enumerate}
 \end{block}
\end{frame}

\begin{frame}
 \frametitle{Integrating OPL/CPLEX in programming sequences}
 \footnotesize
 \begin{block}{Interfaces}
  \begin{itemize}
    \item command line applications, especially \texttt{oplrun}
    \item ILOG Concert-API
    \item CPLEX Callable Library
    \item CPLEX-Interfaces
    \item ILOG Script
  \end{itemize}
 \end{block}
 \pause
 \begin{block}{Selection of applications}
  \begin{itemize}
   \item automated construction and solution of model instances
   \item reading values for decision variables after solution
   \item adding automatically generated data to a model instance
  \end{itemize}
 \end{block}
\end{frame}

\begin{frame}
 \frametitle{Preparations in the cyclic staffing problem example}
 Add the following constraints:
 \begin{gather*}
  \sum_{s\in S}x_s \leq ub\label{eq:CyclicStaffingUB}\\
  \sum_{s\in S}x_s \geq lb\label{eq:CyclicStaffingLB}
 \end{gather*}
 
 If both bounds have the same value, the effective result is:
 \[
  \sum_{s\in S}x_s = ub = lb
 \]
\end{frame}

\subsection{Command line with data sources}
\begin{frame}
 \frametitle{Advantages and disadvantages of the command line application}
 \footnotesize
 \begin{block}{Advantages}
  \begin{itemize}
   \item useable with any programming language, which supports command line applications
   \item quite versatile
   \item well suited for academic research and commercial prototypes
  \end{itemize}
 \end{block}
 \pause
 \begin{block}{Disadvantages}
  \begin{itemize}
   \item has to be combined with given data source interfaces or data input and output has to be coded seperately
   \item susceptible to error in platform configuration
   \item not well suited for productive systems
  \end{itemize}
 \end{block}
\end{frame}

\begin{frame}[fragile]
 \frametitle{Example: VBA}
 \begin{block}{Listings excerpt: command invocation}
  \begin{lstlisting}[language={[Visual]Basic},showstringspaces=false,numbers=none,basicstyle=\scriptsize\ttfamily]
'Solve P1
Shell ("oplrun " & modPath & " " & datPath)
  \end{lstlisting}
 \end{block}
 \begin{block}{Listings excerpt: Workaround for empty solution space}
    \begin{lstlisting}[language={[Visual]Basic},showstringspaces=false]
'Solution not integer
'prepare step 2
dataSheet.Range("Obj").Value = "n"
  \end{lstlisting}
 \end{block}
\end{frame}


\subsection{The Concert-API, example: Java}
\begin{frame}
 \frametitle{Class packages}
 \begin{description}
  \item[ilog.concert] classes providing interfaces to ILOG Concert
  \item[ilog.cplex] classes providing interfaces to ILOG CPLEX
  \item[ilog.cp] classes providing interfaces to ILOG CP Optimizer
  \item[ilog.opl] classes providing interfaces to OPL
 \end{description}
\end{frame}

\begin{frame}
 \frametitle{Important classes}
 \begin{description}
  \item[IloOplFactory] the central class for constructing other OPL objects
  \item[IloOplModelSource] a model file in the file system
  \item[IloOplErrorHandler] pipeline for output of OPL errors and warnings (standard: System.out)
  \item[IloOplModelDefintion] internal representation of the model 
  \item[IloCplex] represents an instance of the CPLEX solver
  \item[IloOplModel] a problem instance
  \item[IloOplDataSource] data source for a model instance
 \end{description}
\end{frame}

\begin{frame}[fragile]
 \frametitle{Instantiating and solving models}
 \begin{block}{Example: Cyclic-Staffing-Problem}
\begin{lstlisting}[language=java,numbers=none,basicstyle=\tiny\ttfamily]
IloOplFactory oplF = new IloOplFactory();

IloOplModelSource modelSource =
  oplF.createOplModelSource("CyclicStaffingProblem.mod");

IloOplErrorHandler err = oplF.createOplErrorHandler();
IloOplModelDefinition def = oplF.createOplModelDefinition(
  modelSource, oplF.createOplSettings(err));

IloCplex cplex = oplF.createCplex();

IloOplModel opl = oplF.createOplModel(def, cplex);

IloOplDataSource dataSource =
 oplF.createOplDataSource("CyclicStaffingProblem.dat");
opl.addDataSource(dataSource);

opl.generate();

opl.getCplex().solve();

opl.printSolution(System.out);
\end{lstlisting}
 \end{block}
\end{frame}

\begin{frame}
 \frametitle{Accesing model elements}
 \begin{itemize}
  \item access with class \structure{\texttt{IloOplElement}} and method \structure{\texttt{getElement(\textsf{String s})}}.
  \item data type translation with \structure{\texttt{as}}-methods
 \end{itemize}
 \begin{table}
 \centering\scriptsize
 \begin{tabular}{*{3}{>{\ttfamily}l}}
  \toprule
  \sffamily\bfseries OPL data type & \sffamily\bfseries Concert data type keyword & \sffamily\bfseries Java data type\\
  \midrule
  %\multicolumn{3}{c}{\cellcolor{gray!10}\footnotesize Primititve Datentypen}\\
  int & {\texttt{Int}} & int\\
  float & {\texttt{Num}} & double\\
  string & {\texttt{Symbol}} & String\\
  \midrule
  %\multicolumn{3}{c}{\cellcolor{gray!10}\footnotesize Abgeleitete Datentypen}\\
  \textrm{Set} & \texttt{Set} & \\
  \textrm{Array} & {\texttt{Map}} & \\
  tuple & {\texttt{Tuple}} & \\
  range & \texttt{Range} & \\
  dvar & {\texttt{Var}} & \\
  dexpr & {\texttt{Expr}} & \\
  \bottomrule
 \end{tabular}
\end{table}
\end{frame}

\begin{frame}[fragile]
 \frametitle{Define custom data sources}
 \begin{itemize}
  \item abstract class \structure{\texttt{IloCustomOplDataSource}}
  \item data delivery via \structure{\texttt{customRead}}
 \end{itemize}
 \begin{block}{Example: cyclic staffing problem}
\begin{lstlisting}[language=java,numbers=none,basicstyle=\tiny\ttfamily]
@Override
public void customRead() {
  //Initialize Datahandler object
  IloOplDataHandler handler = getDataHandler();

  //deliver ub
  handler.startElement("ub");
  handler.addIntItem(this.ub);
  handler.endElement();

  //deliver lb
  handler.startElement("lb");
  handler.addIntItem(this.lb);
  handler.endElement();
}
\end{lstlisting}
 \end{block}
\end{frame}

\subsection{ILOG Script}
\begin{frame}
 \frametitle{ILOG Script}
 \begin{itemize}
  \item ILOG Script is an extension of JavaScript
  \item based on Concert
  \item Scripts are written directly into to the model file
  \item \structure{execute} blocks can be used in pre- and postprocessing
  \begin{itemize}
   \item simplified syntax
   \item OPL variables can be used like script variables
  \end{itemize}
  \item The \structure{main} block serves as procedure control. Here we can create and and solve problem instances.
 \end{itemize}
\end{frame}
