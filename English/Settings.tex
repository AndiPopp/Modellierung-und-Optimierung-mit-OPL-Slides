%Beamer Einstellungen
\usetheme{Goettingen}
\newcounter{chapter}

\setbeamertemplate{section in head/foot}{\hfill\thechapter.\insertsectionheadnumber.~\insertsectionhead}
\setbeamertemplate{section in head/foot shaded}{\color{structure!50}\hfill\thechapter.\insertsectionheadnumber.~\insertsectionhead}
\setbeamertemplate{section in toc}{\large\thechapter.\inserttocsectionnumber~\inserttocsection}
\makeatletter
\setbeamertemplate{section in sidebar}%{sidebar theme}
{%
  \vbox{%
    \vskip1ex%
    \beamer@sidebarformat{3pt}{section in sidebar}{\thechapter.\insertsectionheadnumber
~\insertsectionhead}%
  }%
}
\setbeamertemplate{section in sidebar shaded}%{sidebar theme}
{%
  \vbox{%
    \vskip1ex%
    \beamer@sidebarformat{3pt}{section in sidebar shaded}{\thechapter.\insertsectionheadnumber
~\insertsectionhead}%
  }%
}
\addtobeamertemplate{navigation symbols}{%
	\usebeamerfont{footline}%
 	\usebeamercolor[fg]{footline}%
 	\raisebox{.25ex}{\insertframenumber/\inserttotalframenumber}
}{}
\makeatother

%Farbe
\definecolor{mygreen}{rgb}{0,0.6,0}
\definecolor{mygray}{rgb}{0.5,0.5,0.5}
\definecolor{mymauve}{rgb}{0.58,0,0.82}
\definecolor{mybygrey}{gray}{.93}
\definecolor{eclipse-comments}{rgb}{0,0.6,0}
\definecolor{eclipse-strings}{rgb}{1,0,1}
\definecolor{eclipse-keywords}{rgb}{0,0,1}

% Lenghts
\setlength{\parskip}{.5\baselineskip plus .2\baselineskip minus .1\baselineskip}

%Eurozeichen
\usepackage{eurosym}
\DeclareUnicodeCharacter{20AC}{\euro}

%tabellen
\newcolumntype{L}{>{\raggedright\arraybackslash}X}

%Listings Ilog-Script-Definition
\lstdefinelanguage{IlogScript}{
  keywords=[1]{typeof, new, true, false, catch, function, return, null, catch, switch, var, if, in, while, do, else, case, break, for},
  keywordstyle=[1]\color{orange}\bfseries,
  keywords=[2]{class, export, boolean, throw, implements, import, this, execute, int, main},
  keywordstyle=[2]\color{blue}\bfseries,
  keywords=[3]{write, writeln, thisOplModel, cplex},
  keywordstyle=[3]\color{RedViolet}\bfseries,
  identifierstyle=\color{black},
  sensitive=false,
  comment=[l][\color{eclipse-comments}]{//},%
  morecomment=[s]{/*}{*/},
  commentstyle=\color{purple}\ttfamily,
  stringstyle=\color{mymauve}\ttfamily,
  morestring=[b]',
  morestring=[b]",
  showstringspaces=false
}[keywords,comments,strings]%

%Listings OPL Definition
\lstdefinelanguage[]{opl}%
{
  keywords=[1]{
    maximize, minimize, subject, to, forall, sum, solve, string, dvar, boolean, int, int+, float, float+, enum, tupel, ftoi, mod, abs, maxint, sqrt, ceil, floor, distToInt, frac, trunc, infinity, first, last, card, ord, next, prev, range, in, struct, prod, min, max, union, inter, not, initialize, var, dmin, dmax, dsize, bound, dnexthigher, alldifferent, circuit, distribute, try, endtry, tryall, if, endif, then, else, while, select, once, search, when, onValue, generate, generationMin, generationMax, generateSeq
  },
  keywordstyle=[1]\color{eclipse-keywords},
  morekeywords={..,=,<,>,<=,=>,==},
  keywords=[2]{typeof, new, true, false, catch, function, return, null, catch, switch, var, if, in, while, do, else, case, break},
  keywordstyle=[2]\color{orange}\bfseries,
  keywords=[3]{write, writeln, thisOplModel, from, DBConnection, DBRead, DBUpdate, SheetConnection, SheetRead, SheetWrite},
  keywordstyle=[3]\color{RedViolet}\bfseries,
  comment=[l][\color{eclipse-comments}]{//},%
  morecomment=[s][\color{eclipse-comments}]{/*}{*/},%
  string=[b][\color{eclipse-strings}]\``,%
  stringstyle=\color{mymauve}\ttfamily,
  morestring=[b]",%
  morestring=[b]',%
  showstringspaces=false
}[keywords,comments,strings]%

\lstset{ %
  backgroundcolor=\color{mybygrey},   % choose the background color; you must add \usepackage{color} or \usepackage{xcolor}
  basicstyle=\footnotesize\ttfamily,
  captionpos=b,                    % sets the caption-position to bottom
  commentstyle=\color{mygreen},    % comment style
  %deletekeywords={...},            % if you want to delete keywords from the given language
  escapeinside={\%*}{*)},          % if you want to add LaTeX within your code
  frame=leftline,                    % adds a frame around the code
  keywordstyle=\color{blue},       % keyword style
  language=opl,                 % the language of the code
  linewidth=\textwidth,
  morekeywords={*,documentclass,begin,title,author,date,maketitle},            % if you want to add more keywords to the set
  numbers=left,                    % where to put the line-numbers; possible values are (none, left, right)
  numbersep=5pt,                   % how far the line-numbers are from the code
  numberstyle=\tiny\color{mygray}, % the style that is used for the line-numbers
  rulecolor=\color{black},         % if not set, the frame-color may be changed on line-breaks within not-black text (e.g. comments (green here))
  stepnumber=1,                    % the step between two line-numbers. If it's 1, each line will be numbered
  stringstyle=\color{mymauve},     % string literal style
  tabsize=2,                       % sets default tabsize to 2 spaces
  xleftmargin=.05\textwidth,
  literate=%
    {Ö}{{\"O}}1
    {Ä}{{\"A}}1
    {Ü}{{\"U}}1
    {ß}{{\ss}}1
    {ü}{{\"u}}1
    {ä}{{\"a}}1
    {ö}{{\"o}}1
    {~}{{\textasciitilde}}1,
  title=\lstname                   % show the filename of files included with \lstinputlisting; also try caption instead of title
}

%Listings OPL Data Definition
\lstdefinelanguage[]{opldata}%
{
  keywords=[1]{
    maximize, minimize, subject, forall, sum, solve, string, dvar, boolean, int, int+, float, float+, enum, tupel, ftoi, mod, abs, maxint, sqrt, ceil, floor, distToInt, frac, trunc, infinity, first, last, card, ord, next, prev, range, in, struct, prod, min, max, union, inter, not, initialize, var, dmin, dmax, dsize, bound, dnexthigher, alldifferent, circuit, distribute, try, endtry, tryall, if, endif, then, else, while, select, once, search, when, onValue, generate, generationMin, generationMax, generateSeq
  },
  keywordstyle=[1]\color{eclipse-keywords},
  morekeywords={..,=,<,>,<=,=>,==},
  keywords=[2]{typeof, new, true, false, catch, function, return, null, catch, switch, var, if, in, while, do, else, case, break},
  keywordstyle=[2]\color{orange}\bfseries,
  keywords=[3]{write, writeln, thisOplModel, from, to, DBConnection, DBRead, DBUpdate, SheetConnection, SheetRead, SheetWrite},
  keywordstyle=[3]\color{RedViolet}\bfseries,
  comment=[l][\color{eclipse-comments}]{//},%
  morecomment=[s][\color{eclipse-comments}]{/*}{*/},%
  string=[b][\color{eclipse-strings}]\``,%
  stringstyle=\color{mymauve}\ttfamily,
  morestring=[b]",%
  morestring=[b]',%
  showstringspaces=false
}[keywords,comments,strings]%

\lstset{ %
  backgroundcolor=\color{mybygrey},   % choose the background color; you must add \usepackage{color} or \usepackage{xcolor}
  basicstyle=\footnotesize\ttfamily,
  captionpos=b,                    % sets the caption-position to bottom
  commentstyle=\color{mygreen},    % comment style
  %deletekeywords={...},            % if you want to delete keywords from the given language
  escapeinside={\%*}{*)},          % if you want to add LaTeX within your code
  frame=leftline,                    % adds a frame around the code
  keywordstyle=\color{blue},       % keyword style
  language=opl,                 % the language of the code
  linewidth=\textwidth,
  morekeywords={*,documentclass,begin,title,author,date,maketitle},            % if you want to add more keywords to the set
  numbers=left,                    % where to put the line-numbers; possible values are (none, left, right)
  numbersep=5pt,                   % how far the line-numbers are from the code
  numberstyle=\tiny\color{mygray}, % the style that is used for the line-numbers
  rulecolor=\color{black},         % if not set, the frame-color may be changed on line-breaks within not-black text (e.g. comments (green here))
  stepnumber=1,                    % the step between two line-numbers. If it's 1, each line will be numbered
  stringstyle=\color{mymauve},     % string literal style
  tabsize=2,                       % sets default tabsize to 2 spaces
  xleftmargin=.05\textwidth,
  literate=%
    {Ö}{{\"O}}1
    {Ä}{{\"A}}1
    {Ü}{{\"U}}1
    {ß}{{\ss}}1
    {ü}{{\"u}}1
    {ä}{{\"a}}1
    {ö}{{\"o}}1
    {~}{{\textasciitilde}}1,
  title=\lstname                   % show the filename of files included with \lstinputlisting; also try caption instead of title
}