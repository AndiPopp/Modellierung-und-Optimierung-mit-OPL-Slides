\sectionframe{Errors and warnings in OPL}
\begin{frame}
 \frametitle{Types of errors}
 \begin{block}{Classification by severity}
  \begin{description}
   \item[Error] prevents the succesful completion of the solution run
   \item[Warning] does not prevent the solution run, but can cause unexpected results. Sometimes clue to mistakes in the code.
  \end{description}
 \end{block}
 \begin{block}{Classification by time of occurence}
  \begin{description}
   \item[compiler error] occur during the translation of the problem for the solver. Will be recognized by the IDE.
   \item[runtime error] occur during solver runtime. Will not be recognized by the IDE but displayed after a solution run.
  \end{description}
 \end{block}
\end{frame}

\begin{frame}
 \frametitle{Common error messages during first tries}
 \begin{itemize}
  \item \texttt{syntax errpor, unexpected ...} (compiler error)
  \begin{itemize}
   \item Compiler does not understand the command after "`unexpected"' here
   \item missing semi-colon?
  \end{itemize}
  \item \texttt{syntax errpor, unexpected =} (compiler error)
  \begin{itemize}
   \item special case of above
   \item often mix-up of the assignment operator~\texttt{=} and the comparison operator~\texttt{==}
  \end{itemize}
  \item \texttt{The type ... cannot be used for ...} (Compilerfehler)
  \begin{itemize}
   \item data type mix-up
  \end{itemize}
  \item \texttt{index out of bound for array ...} (Laufzeitfehler)
  \begin{itemize}
   \item tried to access an array with a wrong index value
  \end{itemize}
 \end{itemize}
\end{frame}

