\sectionframe{OPL: custom tupels as data structure}
\begin{frame}
 \frametitle{Example: Relieve Doctors}
 \begin{figure}
  \includegraphics<1>[width=\linewidth,page=1]{Bilder/Graph_Zuordnungsproblem}
  \includegraphics<2>[width=\linewidth,page=2]{Bilder/Graph_Zuordnungsproblem}
 \end{figure}
\end{frame}

\begin{frame}
 \frametitle{Model: Assignment problem}
 \small
 \begin{tabularx}{\linewidth}{lL}
  \multicolumn{2}{l}{\textbf{Index sets}:}\\
     $R$ & set of ressources\\
     $T$ & set of tasks\\
  \multicolumn{2}{l}{\textbf{Parameters}:}\\
     $E$ & set of egdes in the assignment graph\\
     $c_{rt}$ & cost of each edge~$(r, t)\in E$\\
  \multicolumn{2}{l}{\textbf{Decision variables}:}\\
     $x_{rt}$ & binary variable representing the choice of edge~$(r, t)\in E$ \\[1ex]
  \multicolumn{2}{l}{\textbf{Model description}:}\\[1ex]
  \multicolumn{2}{l}{
      $
      \begin{array}{rllr}
	\min & \displaystyle\sum_{(r, t)\in E}  c_{rt}\cdot x_{rt} & & \\[3ex]
	s.t. & \displaystyle\sum_{(r, t)\in E} x_{rt} = 1 & \quad\forall t\in T & \mathrm{(I)}\\
	      & \displaystyle\sum_{(r, t)\in E} x_{rt} \leq 1 & \quad\forall r\in R & \mathrm{(II)}\\
	      & x_{rt} \in \{0, 1\} & \quad\forall (r, t)\in E & \\
      \end{array}
      $
  }\\[1ex]
 \end{tabularx}
\end{frame}


\begin{frame}
 \frametitle{Representation of missing edges in OPL}
 \begin{itemize}
  \item Assign prohibitively high costs to missing edges. Disadvantages:
  \begin{itemize}
   \item unnecessary binary variables
   \item susceptible to machine rounding errors
   \item only applicable to weighted graphs (if at all)
  \end{itemize}
  \item Adjacency matrix. Disadvantages:
  \begin{itemize}
   \item unnecessary binary variables
  \end{itemize}
  \item Adjacency lists. Disadvantages:
  \begin{itemize}
   \item \texttt{x[r in R][t in A[r]]} \textrightarrow{} Error:
    "`Variable indexer size not allowed for a generic array."'
  \end{itemize}
 \end{itemize}
\end{frame}

\shorthandoff{"}
\begin{frame}
 \frametitle{Tuple data structure}
 Tuples are custom data structurs, consisting of elements of other data types.
 
 \begin{block}{Definition of a new tuple data type}
 \small\ttfamily
  tuple \textit{Name\_of\_the\_tuple\_data\_type} \{\\
  \ \ \textit{Data\_type\_of\_1st\_Element} \textit{Name\_of\_1st\_Element};\\
  \ \ \textit{Data\_type\_of\_2nd\_Element} \textit{Name\_of\_2nd\_Element};\\
  \ \ ...\\
  \}
  \end{block}
  
  \begin{block}{Example: Edges as tuple data type}
   \ttfamily\small
   \{string\} V = \{"A", "B", "C"\};\\
   tupel edge \{\\
   \ \ string start;\\
   \ \ string end;\\
  \};
  \end{block}
\end{frame}

\begin{frame}
 \frametitle{Tuple literals and elements}
 In a tuple data type's literals the elements are sorted into angle brackets.
 \begin{block}{Example: Definition of an edge as literal}
  \ttfamily
    edge e = <"A", "B">;
 \end{block}
 
 Single elements of a tuple data type are adressed with a dot.
 \begin{block}{Example: getting the starting vertex of an edge}
  {\ttfamily e.start} \qquad\textrightarrow{}\qquad {\ttfamily "A"}
 \end{block}
\end{frame}
\shorthandon{"}

\begin{frame}
 \frametitle{Application of tuple data type (Alternative~1)}
 Vertices and Edges shall be defined as above.
 
 \begin{block}{Application example}
  $\displaystyle\sum_{(r, t)\in E} x_{rt} = 1  \qquad\forall t\in T$\\
  \begin{center}
   \structure{\textdownarrow{} OPL \textdownarrow{}}
  \end{center}
  {\ttfamily forall(\alert{t} in T)\\
  \quad sum(<r,\alert{t}> in E)(x[<r,t>]) == 1;}
 \end{block}
\end{frame}


