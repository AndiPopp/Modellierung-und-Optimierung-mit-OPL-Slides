\sectionframe{OPL: conditional operators}
\begin{frame}
 \frametitle{Conditional operators}
 Using a colon, we can apply conditions to iteration indexes, which have to be fulfilled for an index to be incorporated by the operator:
 \begin{center}\ttfamily
  sum(\textsf{\textsl{iteration index}} in \textsf{\textsl{index set}} : \textsf{\textsl{condition}})
 \end{center}
 resp. 
 \begin{center}\ttfamily
  forall(\textsf{\textsl{iteration index}} in \textsf{\textsl{index set}} : \textsf{\textsl{condition}})
 \end{center}
 Conditions are logical expressions (not boolean decision variables!)
\end{frame}

\begin{frame}
 \frametitle{Construtction of conditions}
 \begin{block}{Literals for logical values}
  \texttt{true}, \texttt{false}
 \end{block}
 \begin{block}{Comparison operators for logical values}
  \centering\ttfamily
  \begin{tabular}{lcccccc}
    \toprule
    \textrm{math. notation} & $=$ & $\neq$ & $\leq$ & $<$ & $\geq$ & $>$ \\
    \midrule
    \textrm{OPL Syntax} & == & != & <= & < & >= & >\\
    \bottomrule
  \end{tabular}
 \end{block}
 \begin{block}{Logical operators for logical values}
  \centering\ttfamily
  \begin{tabular}{lcccc}
    \toprule
    \textrm{math. notation} & $\neg$ & $\wedge$ & $\vee$ & $\veebar$\\
    \midrule
    \textrm{OPL syntax} & ! & \&\& & || & != \\
    \bottomrule
  \end{tabular}
 \end{block}
\end{frame}

\begin{frame}
 \frametitle{Application of tuple data type (Alternative~2)}
 Vertices and Edges shall be defined as above.
 
 \begin{block}{Application example}
  $\displaystyle\sum_{(r, t)\in E} x_{rt} = 1  \qquad\forall t\in T$\\
  \begin{center}
   \structure{\textdownarrow{} OPL \textdownarrow{}}
  \end{center}
  {\ttfamily forall(\alert{t} in T)\\
  \quad sum(e in E : e.task == \alert{t})(x[e]) == 1;}
 \end{block}
\end{frame}

