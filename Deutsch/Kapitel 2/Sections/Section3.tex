\sectionframe{Mathematische Modelle in OPL-Syntax}
\begin{frame}[fragile]\small
 \frametitle{Beispiel: Produktionsproblem -- Indexmengen}
 \begin{block}{Mathematisches Modell}
  \begin{tabularx}{\linewidth}{lL}
  \multicolumn{2}{l}{\textbf{Indexmengen}:}\\
   $I$& Menge der Produkte\\
   $R$& Menge der Ressourcen\\[1ex]
 \end{tabularx}
 \end{block}\vspace{-1\baselineskip}
 \begin{block}{Modelldatei}\scriptsize
\begin{lstlisting}[numbers=none]
//Indexmengen
{string} I = ...; //Produkte
{string} R = ...; //Ressourcen
\end{lstlisting}
 \end{block}\vspace{-2\baselineskip}
 \begin{block}{Datendatei}\scriptsize
\begin{lstlisting}[numbers=none]
//Indexmengen
I = {"Produkt 1", "Produkt 2", "Produkt 3"}; 
R = {"Maschine A", "Maschine B"}; 
\end{lstlisting}  
 \end{block}
\end{frame}

\begin{frame}[fragile]\small
 \frametitle{Beispiel: Produktionsproblem -- Parameter}
 \begin{block}{Mathematisches Modell}
   \begin{tabularx}{\linewidth}{lL}
    \multicolumn{2}{l}{\textbf{Parameter}:}\\
      $p_i$& Preis von Produkt~$i\in I$\\
      $c_r$& Kapazität von Ressource~$r\in R$\\
      $v_{ri}$& Kapazitätsverbrauch von Produkt~$i\in I$ auf Ressource~$r\in R$\\
   \end{tabularx}
 \end{block}\vspace{-1\baselineskip}
 \begin{block}{Modelldatei}\scriptsize
\begin{lstlisting}[numbers=none]
//Parameter
float p[I] = ...; //Preis
float c[R] = ...; //Kapazität
float v[R][I] = ...; //Kapazitätsverbrauch
\end{lstlisting}
 \end{block}
\end{frame}

\begin{frame}[fragile]\small
 \frametitle{Beispiel: Produktionsproblem -- Parameter}
 \begin{block}{Mathematisches Modell}
   \begin{tabularx}{\linewidth}{lL}
    \multicolumn{2}{l}{\textbf{Parameter}:}\\
      $p_i$& Preis von Produkt~$i\in I$\\
      $c_r$& Kapazität von Ressource~$r\in R$\\
      $v_{ri}$& Kapazitätsverbrauch von Produkt~$i\in I$ auf Ressource~$r\in R$\\
   \end{tabularx}
 \end{block}\vspace{-1\baselineskip}
 \begin{block}{Datendatei}\scriptsize
\begin{lstlisting}[numbers=none]
//Parameter
p = [2.9, 3.3, 2.2];
c = [64.0, 48.0];
v = [
  [5.3, 2.9, 2.5],
  [3.9, 4.8, 3.1]
];
\end{lstlisting} 
 \end{block}
\end{frame}

\begin{frame}[fragile]\small
 \frametitle{Beispiel: Produktionsproblem -- Entscheidungsvariablen}
 \begin{block}{Mathematisches Modell}
  \begin{tabularx}{\linewidth}{lL}
  \multicolumn{2}{l}{\textbf{Entscheidungsvariablen}:}\\
    $x_i$& Produktionsmenge von Produkt $i\in I$\\[2ex]
  \multicolumn{2}{c}{[\ldots]}\\[2ex]
  \multicolumn{2}{l}{ $x_i \geq0\quad\forall i\in I$}\\
  \end{tabularx}\\
 \end{block}
 \begin{block}{Modelldatei}\scriptsize
\begin{lstlisting}[numbers=none]
//Entscheidungsvariablen
dvar float+ x[I]; //Produktionsmenge
\end{lstlisting}
 \end{block}
\end{frame}

\begin{frame}[fragile]\small
 \frametitle{Beispiel: Produktionsproblem -- Zielfunktion}
 \begin{block}{Mathematisches Modell}
  $     \max\quad \displaystyle\sum_{i\in I} p_i\cdot x_i	$
 \end{block}
 \begin{block}{Modelldatei}\scriptsize
\begin{lstlisting}[numbers=none]
//Zielfunktion
maximize sum(i in I)(p[i]*x[i]);
\end{lstlisting}
 \end{block}
\end{frame}

\begin{frame}[fragile]\small
 \frametitle{Beispiel: Produktionsproblem -- Nebenbedingungen}
 \begin{block}{Mathematisches Modell}
  $\text{s.t.}\quad\displaystyle\sum_{i\in I} v_{ri}\cdot x_i \leq c_i  \quad\forall r\in R$
 \end{block}
 \begin{block}{Modelldatei}\scriptsize
\begin{lstlisting}[numbers=none]
//Nebenbedingungen
subject to{
  
  //Kapazitätsrestriktion
  forall(r in R)
    sum(i in I)(v[r,i]*x[i]) <= c[r];
  
}  
\end{lstlisting}
 \end{block}
\end{frame}

\begin{frame}[fragile]
 \frametitle{Beispiel: Produktionsproblem.mod}
 \medskip
\begin{lstlisting}[basicstyle=\scriptsize\ttfamily]
//Indexmengen
{string} I = ...; //Produkte
{string} R = ...; //Ressourcen

//Parameter
float p[I] = ...; //Preis
float c[R] = ...; //Kapazität
float v[R][I] = ...; //Kapazitätsverbrauch

//Entscheidungsvariablen
dvar float+ x[I]; //Produktionsmenge

//Zielfunktion
maximize sum(i in I)(p[i] * x[i]);

//Nebenbedingungen
subject to{
  
  //Kapazitätsrestriktion
  forall(r in R)
    sum(i in I)(v[r][i]*x[i]) <= c[r];
  
}   
\end{lstlisting}
\end{frame}

\begin{frame}[fragile]
 \frametitle{Beispiel: LewigSanstetten.dat}
 \medskip
\begin{lstlisting}[basicstyle=\scriptsize\ttfamily]
//Indexmengen
I = {"Produkt_1", "Produkt_2", "Produkt_3"}; 
R = {"Maschine_A", "Maschine_B"}; 
 
//Parameter
p = [2.9, 3.3, 2.2];
c = [64.0, 48.0];
v = [
  [5.3, 2.9, 2.5],
  [3.9, 4.8, 3.1]
]; 
\end{lstlisting}
\end{frame}

\begin{frame}
 \frametitle{Lösung der Modellinstanz}
 \ttfamily
 > oplrun -v Produktionsproblem.mod LewigSanstetten.dat
 \begin{center}
  \textsf ...
 \end{center}
 OBJECTIVE: 35.61677 \hfill\alert{\textsf{\textleftarrow{} Optimalwert}}
 \begin{center}
  \textsf ...
 \end{center}
 x = [11.737 0 0.71856]; \hfill\alert{\textsf{\textleftarrow{} Optimallösung}}
\end{frame}
