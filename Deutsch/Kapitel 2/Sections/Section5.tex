\sectionframe{Fehler und Warnungen in OPL}
\begin{frame}
 \frametitle{Arten von Fehlern}
 \begin{block}{Einteilung nach Schwere}
  \begin{description}
   \item[Fehler] Verhindert das erfolgreiche Abschließen eines Lösungslaufs
   \item[Warnung] Behindert den Lösungslauf nicht, es kann aber zu unerwünschten Ergebnissen kommen. Manchmal Hinweis auf Fehler im Code.
  \end{description}
 \end{block}
 \begin{block}{Einteilung nach Zeitpunkt des Auftretens}
  \begin{description}
   \item[Compilerfehler] Treten bei Übersetzung des Problems für den Solver auf. Werden von der IDE erkannt.
   \item[Laufzeitfehler] Treten erst zur Laufzeit des Solvers auf. Werden nicht von der IDE erkannt, aber nach Lösungslauf angezeigt.
  \end{description}
 \end{block}
\end{frame}

\begin{frame}
 \frametitle{Häufige Fehlermeldungen bei ersten Versuchen}
 \begin{itemize}
  \item \texttt{syntax error, unexpected ...} (Compilerfehler)
  \begin{itemize}
   \item Compiler versteht die Anweisung nach "`unexpected"' an dieser Stelle nicht
   \item fehlender Strichpunkt?
  \end{itemize}
  \item \texttt{syntax error, unexpected =} (Compilerfehler)
  \begin{itemize}
   \item Spezialfall zu oben
   \item meist Verwechslung von Zuweisungsoperator~\texttt{=} und Vergleichsoperator~\texttt{==}
  \end{itemize}
  \item \texttt{Der Typ ... kann nicht für ... verwendet werden} (Compilerfehler)
  \begin{itemize}
   \item Datentypen durcheinander gebracht
  \end{itemize}
  \item \texttt{Der Index für den Array ... liegt außerhalb des gültigen Bereichs} (Laufzeitfehler)
  \begin{itemize}
   \item ein Array wurde mit einem ungültigen Index angesprochen
  \end{itemize}
 \end{itemize}
\end{frame}

