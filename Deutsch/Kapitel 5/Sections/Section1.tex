\sectionframe{Weiche Nebenbedingungen}
\begin{frame}
 \frametitle{Beispiel: Produktionsproblem}
 \begin{equation*}
  \begin{array}{rllr}
    \max & \displaystyle\sum_{i\in I} p_i\cdot x_i & & \\[3ex]
    s.t. & \displaystyle\sum_{i\in I} v_{ri}\cdot x_i \leq c_i & \quad\forall r\in R & \mathrm{(\alert{I})}\\
	  & x_i \geq 0 & \quad\forall i\in I & \\
  \end{array}
 \end{equation*}
 
 Nebenbedingung~\textrm{(\alert{I})} ist eine "`harte"' Nebenbedingung, sie muss exakt eingehalten werden.
\end{frame}

\begin{frame}
 \frametitle{Weiche Ungleichungsbedingung}
 \begin{equation*}
  \begin{array}{rllr}
    \max & \displaystyle\sum_{i\in I} p_i\cdot x_i & & \\[3ex]
    s.t. & \displaystyle\sum_{i\in I} v_{ri}\cdot x_i \leq c_r + \colorbox{gray!20}{$o_r$} & \quad\forall r\in R & \mathrm{(I)}\\
	  & x_i, \colorbox{gray!20}{$o_r$} \geq 0 & \quad\forall i\in I, r\in R & \\
  \end{array}
 \end{equation*}
 
 Problem: keine Optimallösung, da Lösungsraum in Optimierungsrichtung unbeschränkt
\end{frame}

\begin{frame}
 \frametitle{Weiche Ungleichungsbedingung mit Strafkosten und Beschränkung}
 \begin{equation*}
  \begin{array}{rllr}
    \max & \displaystyle\sum_{i\in I} p_i\cdot x_i - \sum_{r\in R} \colorbox{gray!20}{$k_r$} \cdot o_r & & \\[3ex]
    s.t. & \displaystyle\sum_{i\in I} v_{ri}\cdot x_i \leq c_r + o_r & \quad\forall r\in R & \mathrm{(I)}\\
	& o_r \leq \colorbox{gray!20}{$m_r$} & \quad\forall r\in R & \mathrm{(II)}\\[1ex]
	& x_i,o_r \geq 0 & \quad\forall i\in I, r\in R & \\
  \end{array}
 \end{equation*}  
\end{frame}

\begin{frame}
 \frametitle{Beispiel: Produktionsproblem mit Vollauslastung}
 \begin{equation*}
  \begin{array}{rllr}
    \max & \displaystyle\sum_{i\in I} p_i\cdot x_i & & \\[3ex]
    s.t. & \displaystyle\sum_{i\in I} v_{ri}\cdot x_i = c_i & \quad\forall r\in R & \mathrm{(\alert{I})}\\
	  & x_i \geq 0 & \quad\forall i\in I & \\
  \end{array}
 \end{equation*}
 
 Nebenbedingung~\textrm{(\alert{I})} ist eine "`harte"' Nebenbedingung, sie muss exakt eingehalten werden.
\end{frame}

\begin{frame}
 \frametitle{Weiche Gleichhungsbedingung}
  \begin{equation*}
    \begin{array}{rllr}
      \max & \displaystyle\sum_{i\in I} p_i\cdot x_i - \sum_{r\in R} \colorbox{gray!20}{$k_r$} \cdot \colorbox{gray!20}{$|o_r|$} & & \\[3ex]
      s.t. & \displaystyle\sum_{i\in I} v_{ri}\cdot x_i = c_i + \colorbox{gray!20}{$o_r$} & \quad\forall r\in R & \mathrm{(I)}\\
	    & x_i \geq 0, \colorbox{gray!20}{$o_r$} \lessgtr0 & \quad\forall i\in I, r\in R & \\
    \end{array}
  \end{equation*}
 
 Problem: Absolutbetrag ist keine lineare Funktion.
\end{frame}

\begin{frame}
 \frametitle{Weiche Gleichhungsbedingung}
  Lösung: Substituiere $o_r = o_r^+ - o_r^-$
  
  \begin{equation*}
    \begin{array}{rllr}
      \max & \displaystyle\sum_{i\in I} p_i\cdot x_i - \sum_{r\in R} k_r \cdot \colorbox{gray!20}{$o_r^+ + o_r^-$} & & \\[3ex]
      s.t. & \displaystyle\sum_{i\in I} v_{ri}\cdot x_i = c_i + \colorbox{gray!20}{$o_r^+ - o_r^-$} & \quad\forall r\in R & \mathrm{(I)}\\
	    & x_i, \colorbox{gray!20}{$o_r^+$}, \colorbox{gray!20}{$o_r^-$} \geq 0 & \quad\forall i\in I, r\in R & \\
    \end{array}
  \end{equation*}
  
  \begin{alertblock}{Vorsicht bei Auflösung von Absolutbeträgen}
   Die Zerlegung einer Variable in zwei Summanden ist nicht eindeutig. Es muss sichergestellt sein, dass ein Summand null ist.
  \end{alertblock}
\end{frame}

