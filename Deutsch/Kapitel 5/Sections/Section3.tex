\sectionframe{Mehrere Zielfunktionen und Pareto-Optimalität}
\begin{frame}
 \frametitle{Beispiel: Lewbrandt GmbH}
 Gesamtkapazität: $120\,$h\\
 \begin{center}\footnotesize
  \begin{tabular}{lccccc}
  \toprule
  \bfseries Auftrag & \bfseries 1 & \bfseries 2 & \bfseries 3 & \bfseries 4 & \bfseries 5 \\
  \midrule
  Deckungsbeitrag & $150\,$k€ & $100\,$k€ & $150\,$k€ & $50\,$k€ & $70\,$k€\\
  Umsatz & $340\,$k€ & $190\,$k€ & $220\,$k€ & $85\,$k€ & $215\,$k€ \\
  Abwasser & $6,2\,$t & $3,5\,$t & $5,8\,$t & $2,4\,$t & $4,8\,$t \\
  Kapazitätsverbrauch &  $65\,$h & $35\,$h & $65\,$h & $15\,$h & $25\,$h\\
  \bottomrule
  \end{tabular}
 \end{center}
 
 Welche Aufträge sollen gefertigt werden? \\\textrightarrow{} Rucksackproblem
 
 \begin{block}{Problem}
  Es gibt drei Zielfunktionen, damit gibt es keine eindeutige Optimallösung.
 \end{block}
\end{frame}

\begin{frame}
 \frametitle{Pareto-Optimalität}
 \begin{block}{Definition: Pareto-Optimalität}
  Eine Lösung heißt pareto-optimal, wenn es keine andere Lösung gibt, die in einer Zielgröße besser ist und in den anderen mindestens gleich gut. 
 \end{block}
 
 \begin{block}{Ausgewählte Lösungen des Beispiels "`Lewbrandt GmbH"'}
  \footnotesize
 \centering
 \begin{tabular}{*{5}{c}rrrc}
  \toprule
  $x_1$ & $x_2$ & $x_3$ & $x_4$ & $x_5$ & \scriptsize Gewinn & \scriptsize Umsatz & \scriptsize Abwasser & p.-o. \\
  \midrule
  0&	1&	0&	1&	0&	150&	275&	5,9&	ja\\
  \alert{0}&	\alert{1}&	\alert{0}&	\alert{1}&	\alert{1}&	\alert{220}&	\alert{490}&	\alert{10,7}&	\alert{nein}\\
  1&	1&	0&	0&	0&	250&	530&	9,7&	ja\\
  1&	1&	0&	1&	0&	300&	615&	12,1&	ja\\
  \bottomrule
 \end{tabular}
 \end{block}

\end{frame}

