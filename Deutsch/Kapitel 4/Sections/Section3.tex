\sectionframe{OPL: Selbst\-de\-fi\-nier\-te Tupel als Da\-ten\-struk\-tur}
\begin{frame}
 \frametitle{Beispiel: Relieve Ärzte}
 \begin{figure}
  \includegraphics<1>[width=\linewidth,page=1]{Bilder/Graph_Zuordnungsproblem}
  \includegraphics<2>[width=\linewidth,page=2]{Bilder/Graph_Zuordnungsproblem}
 \end{figure}
\end{frame}

\begin{frame}
 \frametitle{Modell: Zuordnungsproblem}
 \small
 \begin{tabularx}{\linewidth}{lL}
  \multicolumn{2}{l}{\textbf{Indexmengen}:}\\
     $R$ & Menge der Ressourcen\\
     $T$ & Menge der Aufgaben\\
  \multicolumn{2}{l}{\textbf{Parameter}:}\\
     $E$ & Menge der Kanten im Zuordnungsgraphen\\
     $c_{rt}$ & Kosten für die Auswahl der Kante $(r, t)\in E$\\
  \multicolumn{2}{l}{\textbf{Entscheidungsvariablen}:}\\
     $x_{rt}$ & Binärvariable, die angibt ob die Kante $(r, t)\in E$ ausgewählt wurde\\[1ex]
  \multicolumn{2}{l}{\textbf{Modellbeschreibung}:}\\[1ex]
  \multicolumn{2}{l}{
      $
      \begin{array}{rllr}
	\min & \displaystyle\sum_{(r, t)\in E}  c_{rt}\cdot x_{rt} & & \\[3ex]
	s.t. & \displaystyle\sum_{(r, t)\in E} x_{rt} = 1 & \quad\forall t\in T & \mathrm{(I)}\\
	      & \displaystyle\sum_{(r, t)\in E} x_{rt} \leq 1 & \quad\forall r\in R & \mathrm{(II)}\\
	      & x_{rt} \in \{0, 1\} & \quad\forall (r, t)\in E & \\
      \end{array}
      $
  }\\[1ex]
 \end{tabularx}
\end{frame}


\begin{frame}
 \frametitle{Abbildung der fehlenden Kanten in OPL}
 \begin{itemize}
  \item Fehlende Kanten mit prohibitiv hohen Kosten versehen. Nachteile:
  \begin{itemize}
   \item überflüssige Binärvariablen
   \item anfällig für Maschinenrundungsfehler
   \item wenn dann nur in gewichteten Graphen möglich
  \end{itemize}
  \item Adjazenzmatrix. Nachteile:
  \begin{itemize}
   \item überflüssige Binärvariablen
  \end{itemize}
  \item Adjazenzlisten. Nachteile:
  \begin{itemize}
   \item \texttt{x[r in R][t in A[r]]} \textrightarrow{} Fehlermeldung:
    "`Die Größe des Variablenindexers ist für einen generischen Array nicht zulässig."'
  \end{itemize}
 \end{itemize}
\end{frame}

\shorthandoff{"}
\begin{frame}
 \frametitle{Datenstruktur Tupel}
 Tupel sind selbstdefinierte Datenstrukturen, die aus Elementen anderer Datenstrukturen bestehen.
 
 \begin{block}{Definition eines neuen Tupel-Datentyps}
 \small\ttfamily
  tuple \textit{Name\_der\_Tupel-Datenstruktur} \{\\
  \ \ \textit{Datentyp\_des\_1.\_Elements} \textit{Name\_des\_1.\_Elements};\\
  \ \ \textit{Datentyp\_des\_2.\_Elements} \textit{Name\_des\_2.\_Elements};\\
  \ \ ...\\
  \}
  \end{block}
  
  \begin{block}{Beispiel: Kanten als Tupel-Datentyp}
   \ttfamily\small
   \{string\} V = \{"A", "B", "C"\};\\
   tupel edge \{\\
   \ \ string start;\\
   \ \ string end;\\
  \};
  \end{block}
\end{frame}

\begin{frame}
 \frametitle{Tupel-Literale und -Elemente}
 In Literalen eines Tupel-Datentyps werden die Elemente der Reihenfolge nach in spitzen Klammern zugeordnet.
 \begin{block}{Beispiel: Definition einer Kante als Literal}
  \ttfamily
    edge e = <"A", "B">;
 \end{block}
 
 Einzelne Elemente eines Tupel-Datentyps werden mit einem Punkt angesprochen.
 \begin{block}{Beispiel: Auslesen des Startknotens einer Kante}
  {\ttfamily e.start} \qquad\textrightarrow{}\qquad {\ttfamily "A"}
 \end{block}
\end{frame}
\shorthandon{"}

\begin{frame}
 \frametitle{Anwendung von Tupel-Datentypen (Variante 1)}
 Knoten und Kanten seien wie oben definiert.
 
 \begin{block}{Anwendungsbeispiel}
  $\displaystyle\sum_{(r, t)\in E} x_{rt} = 1  \qquad\forall t\in T$\\
  \begin{center}
   \structure{\textdownarrow{} OPL \textdownarrow{}}
  \end{center}
  {\ttfamily forall(\alert{t} in T)\\
  \quad sum(<r,\alert{t}> in E)(x[<r,t>]) == 1;}
 \end{block}
\end{frame}


