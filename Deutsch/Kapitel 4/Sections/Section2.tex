\sectionframe{Abbilden von Graphen in OPL}
\begin{frame}
 \frametitle{Reihenfolgeabhängiges Produktionsproblem}
 \footnotesize
 \begin{tabularx}{\linewidth}{lL}
  \multicolumn{2}{l}{\textbf{Indexmengen}:}\\
     $I$ & Menge der Produkte\\
     $R$ & Menge der Ressourcen\\
  \multicolumn{2}{l}{\textbf{Parameter}:}\\
     $p_i$ & Preis von Produkt $i\in I$\\
     $c_r$ & Kapazität von Ressource $r\in R$\\
     $v_{ri}$ & Kapazitätsverbrauch von Produkt $i\in I$ auf Ressource $r\in R$\\
     $E$ & Menge der Kanten im Reihenfolgegraph\\
  \multicolumn{2}{l}{\textbf{Entscheidungsvariablen}:}\\
     $x_i$ & Produktionsmenge von Produkt $i\in I$\\[1ex]
  \multicolumn{2}{l}{\textbf{Modellbeschreibung}:}\\[1ex]
  \multicolumn{2}{l}{
      $
      \begin{array}{rllr}
	\max & \displaystyle\sum_{i\in I} p_i\cdot x_i & & \\[3ex]
	s.t. & \displaystyle\sum_{i\in I} v_{ri}\cdot x_i \leq c_i & \quad\forall r\in R & \mathrm{(I)}\\
	      & x_i \geq\displaystyle\sum_{(i, j)\in E}x_j &\quad\forall i\in I & \mathrm{(II)}\\
	      & x_i \geq 0 & \quad\forall i\in I & \\
      \end{array}
      $
  }\\[1ex]
 \end{tabularx}
\end{frame}


\begin{frame}
 \frametitle{Graphen in Optimierungsproblemen}
  \begin{block}{Anwendungsbeispiel für Graphen}
    \[
      x_i \geq \sum_{\alert{(i, j) \in E}}x_j\qquad\forall i\in I  
    \]
  \end{block}
  Frage: Wie kann der Graph in einem Optimierungsmodell abgebildet werden?
\end{frame}

\begin{frame}
 \frametitle{Die Adjazenzmatrix}
 \begin{block}{Definition: Adjazenzmatrix}
  Die Adjazenzmatrix eines Graphen $G=(V, E)$ mit $V=\{V_1, \ldots, V_N\}$ ist eine quadratische $N\times N$-Matrix $(a_{ij})$, für die gilt:
  \begin{equation}
    a_{ij} = \left\{\begin{array}{ll}
		      1 & \text{Kante } V_i\rightarrow V_j \text{ existiert}\\
		      0 & \text{sonst}
		    \end{array}\right.
   \end{equation}
 \end{block}
\end{frame}

\begin{frame}
 \frametitle{Adjazenzmatrix im Beispiel}
  \begin{center}
    \includegraphics<1>[width=.4\linewidth,page=1]{Bilder/Graph_Lewig_Adelburg}\\[2ex]
    \structure{\textdownarrow{} Übersetzung in Adjazenzmatrix \textdownarrow{}}\\[2ex]
    $
    \bordermatrix{
	& I_1 & I_2 & J_1 & J_2 & J_3 & J_4 \cr
    I_1 & 0   & 0   & 1   & 1   & 0   & 0 \cr
    I_2 & 0   & 0   & 0   & 0   & 1   & 0 \cr
    J_1 & 0   & 0   & 0   & 0   & 0   & 0 \cr
    J_2 & 0   & 0   & 0   & 0   & 0   & 1 \cr
    J_3 & 0   & 0   & 0   & 0   & 0   & 0 \cr
    J_4 & 0   & 0   & 0   & 0   & 0   & 0 \cr
    }
    $
  \end{center}
\end{frame}

\begin{frame}[fragile]
 \frametitle{Anwendung von Adjazenzmatrizen in Optimierungsproblemen}
\begin{lstlisting}[numbers=none,basicstyle=\scriptsize\ttfamily]
{string} I = ...;
int a [I,I] = [
  [0, 0, 1, 1, 0, 0],
  [0, 0, 0, 0, 1, 0],
  [0, 0, 0, 0, 0, 0],
  [0, 0, 0, 0, 0, 1],
  [0, 0, 0, 0, 0, 0],
  [0, 0, 0, 0, 0, 0],
];
\end{lstlisting}
\vspace{-\baselineskip}
\[
  x_i \geq \sum_{\alert{(i, j) \in E}}x_j\qquad\forall i\in I  
\]
\begin{center}
 \structure{\textdownarrow{} OPL \textdownarrow{}}
\end{center}
\begin{lstlisting}[numbers=none]
forall(i in I)
  x[i] >= sum (j in I)(a[i,j]*x[j]);
\end{lstlisting}
\end{frame}

\begin{frame}
 \frametitle{Adjazenzlisten}
 \begin{block}{Definition: Adjazenzliste}
  Die Adjazenzliste eines Knoten~$v\in V$ eines Graphen $G=(V, E)$ ist eine Menge $A_v\subseteq V$, welche alle Nachfolger von $v$ beinhaltet.
 \end{block}
 \begin{block}{Adjazenzlisten im Beispiel}
  \includegraphics<1>[width=.4\linewidth,page=1]{Bilder/Graph_Lewig_Adelburg}\quad
  \raisebox{6ex}{
    \begin{minipage}{.5\textwidth}
    \begin{tabular}{ll}
      $A_{I_1} = \{J_1, J_2\}$  & $A_{I_2} = \{J_3\}$\\
      $A_{J_1} = \{\}$ & $A_{J_2} = \{J_4\}$\\
      $A_{J_3} = \{\}$ & $A_{J_4} = \{\}$
    \end{tabular}
    \end{minipage}}
 \end{block}
\end{frame}

\begin{frame}[fragile]
 \frametitle{Anwendung von Adjazenzlisten in Optimierungsproblemen}
 \bigskip
\begin{lstlisting}[numbers=none,basicstyle=\scriptsize\ttfamily]
{string} I = ...;
{string} A[I] = [
  {"J1", "J2"},
  {"J3"},
  {},
  {"J4"},
  {},
  {}
];
\end{lstlisting}
\vspace{-\baselineskip}
\[
  x_i \geq \sum_{\alert{(i, j) \in E}}x_j\qquad\forall i\in I  
\]
\begin{center}
 \structure{\textdownarrow{} OPL \textdownarrow{}}
\end{center}
\begin{lstlisting}[numbers=none]
forall (i in I)
  x[i] >= sum(j in A[i])(x[j]);
\end{lstlisting}
\end{frame}

