\sectionframe{OPL: Operatoren mit Bedingungen}
\begin{frame}
 \frametitle{Operatoren mit Bedingungen}
 Mithilfe eines Doppelpunkts lassen sich Bedingungen an Laufindizes stellen, die erfüllt werden müssen, damit der Index vom Operator berücksichtigt wird:
 \begin{center}\ttfamily
  sum(\textsf{\textsl{Laufindex}} in \textsf{\textsl{Indexmenge}} : \textsf{\textsl{Bedingung}})
 \end{center}
 bzw. 
 \begin{center}\ttfamily
  forall(\textsf{\textsl{Laufindex}} in \textsf{\textsl{Indexmenge}} : \textsf{\textsl{Bedingung}})
 \end{center}
 Bedingungen sind logische Ausdrücke (keine Boolean-Entscheidungsvariablen!)
\end{frame}

\begin{frame}
 \frametitle{Konstruktion von Bedingungen}
 \begin{block}{Literale für Wahrheitswerte}
  \texttt{true}, \texttt{false}
 \end{block}
 \begin{block}{Vergleichsoperatoren für Wahrheitswerte}
  \centering\ttfamily
  \begin{tabular}{lcccccc}
    \toprule
    \textrm{math. Schreibweise} & $=$ & $\neq$ & $\leq$ & $<$ & $\geq$ & $>$ \\
    \midrule
    \textrm{OPL-Syntax} & == & != & <= & < & >= & >\\
    \bottomrule
  \end{tabular}
 \end{block}
 \begin{block}{Logische Verknüpfungen für Wahrheitswerte}
  \centering\ttfamily
  \begin{tabular}{lcccc}
    \toprule
    \textrm{math. Schreibweise} & $\neg$ & $\wedge$ & $\vee$ & $\veebar$\\
    \midrule
    \textrm{OPL-Syntax} & ! & \&\& & || & != \\
    \bottomrule
  \end{tabular}
 \end{block}
\end{frame}

\begin{frame}
 \frametitle{Anwendung von Tupel-Datentypen (Variante 2)}
 Knoten und Kanten seien wie oben definiert.
 
 \begin{block}{Anwendungsbeispiel}
  $\displaystyle\sum_{(r, t)\in E} x_{rt} = 1  \qquad\forall t\in T$\\
  \begin{center}
   \structure{\textdownarrow{} OPL \textdownarrow{}}
  \end{center}
  {\ttfamily forall(\alert{t} in T)\\
  \quad sum(e in E : e.task == \alert{t})(x[e]) == 1;}
 \end{block}
\end{frame}

