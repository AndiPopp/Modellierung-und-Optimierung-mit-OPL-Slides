\sectionframe{Lineare Optimierung}
\subsection{Eigenschaften}
\begin{frame}\small
 \frametitle{Lineare Funktionen und Nebenbedingungen}
 \begin{block}{Lineare Funktion}
  \vspace{-2\baselineskip}
  \begin{equation*}
    f(x_1, \ldots, x_N) = \sum_{n=1}^{N} c_n\cdot x_n
  \end{equation*}
 \end{block}
 \vspace{-\baselineskip}
 \begin{block}{Lineare Nebenbedingung}
  Sei $f$ eine lineare Funktion:
  \begin{align*}
   &f(x_1, \ldots, x_N) = b\\
   &f(x_1, \ldots, x_N) \leq b\\
   &f(x_1, \ldots, x_N) \geq b\\
  \end{align*}
 \end{block}
 \vspace{-2\baselineskip}
 \begin{block}{Lineares Optimierungsmodell}
  Zielfunktion und Nebenbedingung linear in den Entscheidungsvariablen $\Longrightarrow$ lineares Optimierungsmodell
 \end{block}
\end{frame}

\begin{frame}
 \frametitle{Eigenschaften linearer Funktionen}
 \begin{description}
  \item[Proportionalität] Jede Variable trägt einen proportionalen Wert zur Funktion bei.
  \item[Unabhängigkeit] Der Wert, den eine Variable zur Funktion beiträgt ist unabhängig
von der Ausprägung der anderen Variablen.
 \end{description}

\end{frame}


\begin{frame}
 \frametitle{Typische Anzeichen für Nichtlinearität}
 \begin{itemize}
  \item Variablen haben einen anderen Exponenten als~$1$
  \begin{itemize}
   \item andere natürliche Exponenten, z.B.: $x^2$
   \item Wurzlen, z.B.: $\sqrt{x} = x^{\frac{1}{2}}$
   \item Variablen im Nenner, z.B.: $\frac{1}{x} = x^{-1}$
  \end{itemize}
  \item Variablen werden miteinander multipliziert, z.B.: $x_1\cdot x_2$
  \item Exponentialfunktionen, z.B.: $2^x$
  \item Absolutbeträge, z.B. $|x|$
 \end{itemize}

 \begin{block}{Besonderheit: Konstanten}
  Konstanten sind grundsätzlich nicht linear, stören aber in linearen Optimierungsmodellen nicht, da sie stets auflösbar sind.
 \end{block}
\end{frame}

\begin{frame}
 \frametitle{deterministische vs. stochastische Optimierungsmodelle}
 \begin{description}
  \item[deterministische Optimierungsmodelle:] alle Paramter und Funktionswerte sind stets eindeutig bekannt
  \item[stochastische Optimierungsmodelle:] Parameter und Funktionswerte unterliegen zufälligen Schwankung
 \end{description}
 
 Lineare Optimierungsmodelle sind grundsätzlich deterministisch.
\end{frame}

\begin{frame}
 \frametitle{kontinuierliche vs. ganzzahlige Optimierungsmodelle}
 \begin{description}
  \item[kontinuierliche Optimierungsmodelle:] die Werte der Entscheidungsvariablen sind beliebig teilbar (reelle Werte)
  \item[ganzzahlige Optimierungsmodelle:] die Werte der Entscheidungsvariablen können nur ganzzahlige Werte annehmen
 \end{description}
 
 \begin{block}{Arten von linearen Optimierungsmodellen nach zulässigen Werten für Entscheidungsvariablen:}
  \begin{itemize}\footnotesize
   \item kontinuierliche Entscheidungsvariablen $\Longrightarrow$ (kontinuierliches) lineares Optimierungsmodell
   \item ganzzahlige Entscheidungsvariablen $\Longrightarrow$ ganzzahliges lineares Optimierungsmodell
   \item sowohl kontinuierliche als auch ganzzahlige Entscheidungsvariablen $\Longrightarrow$ gemischt-ganzzahliges lineares Optimierungsmodell
  \end{itemize}
 \end{block}
\end{frame}

\subsection{Lösung von linearen Modellen}

\begin{frame}
 \frametitle{Lösungsstrukturen für lineare Optimierungsmodelle}
 \begin{itemize}
  \item es gibt genau eine Optimallösung
  \item es gibt unendlich viele Optimallösung %(jede Konvexkombination von zwei Optimallösungen ist wieder eine Optimallösung)
  \item es gibt keine Optimallösung
  \begin{itemize}
   \item der Lösungsraum ist leer
   \item der Lösungsraum ist unbeschränkt und die Zielfunktion geht gegen unendlich
  \end{itemize}
 \end{itemize}
\end{frame}


\begin{frame}
 \frametitle{Gängige Lösungsverfahren für lineare Optimierungsmodelle}
 \begin{block}{Lösungsverfahren für (kontinuierliche) lineare Optimierungsmodelle}
  \begin{itemize}
    \item Simplex-Verfahren von Dantzig
    \item Innere-Punkte-Verfahren von Karmarkar 
  \end{itemize}
 \end{block}
 \begin{block}{Lösungsverfahren für (gemischt-)ganzzahlige lineare Optimierungsmodelle}
  \begin{itemize}
    \item Branch-and-Bound-Verfahren
    \item Schnittebenenverfahren 
  \end{itemize}
 \end{block}
\end{frame}




