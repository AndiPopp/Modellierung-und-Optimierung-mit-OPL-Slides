\sectionframe{Modell und Modellinstanz}
\begin{frame}
 \frametitle{Optimierungsmodell aus Beispiel ``Lewig Sanstetten''}
 \begin{block}{Modell: Produktionsproblem}
 \Large
 \begin{equation*}
  \begin{array}{ll}
    \max & 2,9\cdot x_1 + 3,3\cdot x_2 + 2,2\cdot x_3\\
    s.t. & 5,3\cdot x_1 + 2,9\cdot x_2 + 2,5\cdot x_3 \leq 64\\
	  & 3,9\cdot x_1 + 4,8\cdot x_2 + 3,1\cdot x_3 \leq 48\\
	  & x_1\geq0,\,x_2\geq0,\,x_3\geq0\\
  \end{array}
 \end{equation*}
 \end{block}
\end{frame}

\begin{frame}
 \frametitle{Ersetzen von Daten durch Parameter (Formvariablen)}
 \begin{block}{Modell: Produktionsproblem}
 \Large
 \begin{equation*}
  \begin{array}{ll}
  \max & p_1\cdot x_1 + p_2\cdot x_2 + p_3\cdot x_3\\
  s.t. & v_{A1}\cdot x_1 + v_{A2}\cdot x_2 + v_{A3}\cdot x_3 \leq c_A\\
	& v_{B1}\cdot x_1 + v_{B2}\cdot x_2 + v_{B3}\cdot x_3 \leq c_B\\
	& x_1\geq0,\,x_2\geq0,\,x_3\geq0\\
  \end{array}
 \end{equation*}
 \end{block}
\end{frame}

\begin{frame}
 \frametitle{Indexmengen als spezielle Parameter}
 \begin{block}{Variante 1: Maximalindex als Parameter}
  Sei $I\in\mathbb{N}$ die Anzahl der Produkte, dann lautet zum Beispiel die Zielfunktion:
  \[
   \sum_{i=1}^I p_i \cdot x_i
  \]
 \end{block}
 \begin{block}{Variante 2: Indexmenge als Parameter}
  Sei $I$ die Menge der Produkte, dann lautet zum Beispiel die Zielfunktion:
  \[
   \sum_{i\in I} p_i\cdot x_i
  \]
 \end{block}
\end{frame}

\begin{frame}
 \frametitle{Summation über parametrisierte Indexmengen}
 \begin{block}{Modell: Produktionsproblem}
 \Large
 \begin{equation*}
  \begin{array}{ll}
  \max & \displaystyle\sum_{i\in I} p_i\cdot x_i\\
  s.t. & \displaystyle\sum_{i\in I} v_{Ai}\cdot x_i \leq c_A\\
	& \displaystyle\sum_{i\in I} v_{Bi}\cdot x_i \leq c_B\\
	& x_1\geq0,\,x_2\geq0,\,x_3\geq0\\
  \end{array}
 \end{equation*}
 \end{block}
\end{frame}

\begin{frame}
 \frametitle{Verwendung des Allquantors}
 \Large
 \begin{align*}
  & \sum_{i\in I} v_{Ai}\cdot x_i \leq c_A\\
  & \sum_{i\in I} v_{Bi}\cdot x_i \leq c_B
 \end{align*}
 \begin{center}\normalsize
  \structure{\textdownarrow{} Indexmenge $R$ der Ressourcen \textdownarrow{}}
 \end{center}
 \begin{equation*}
  \sum_{i\in I} v_{ri}\cdot x_i \leq c_r\qquad\alert{\forall r\in R}
 \end{equation*}
\end{frame}

\begin{frame}
 \frametitle{Allgemeines Modell}
 \begin{block}{Modell: Produktionsproblem}
 \begin{equation*}
  \begin{array}{lll}
  \max & \displaystyle\sum_{i\in I} p_i\cdot x_i\\
  s.t. & \displaystyle\sum_{i\in I} v_{ri}\cdot x_i \leq c_r&\qquad\forall r\in R \\
	& x_i\geq0 &\qquad\forall i\in I\\
  \end{array}
 \end{equation*}
 \end{block}
\end{frame}

\begin{frame}
 \frametitle{Modell: Produktionsproblem}
 \footnotesize
 \begin{tabularx}{\linewidth}{lL}
  \multicolumn{2}{l}{\textbf{Indexmengen}:}\\
   $I$& Menge der Produkte\\
   $R$& Menge der Ressourcen\\[1ex]
  \multicolumn{2}{l}{\textbf{Parameter}:}\\
   $p_i$& Preis von Produkt $i\in I$\\
   $c_r$& Kapazität von Ressource $r\in R$\\
   $v_{ri}$& Kapazitätsverbrauch von Produkt $i\in I$ auf Ressource $r\in R$\\[1ex]
  \multicolumn{2}{l}{\textbf{Entscheidungsvariablen}:}\\
   $x_i$& Produktionsmenge von Produkt $i\in I$\\[1ex]
  \multicolumn{2}{l}{\textbf{Modellbeschreibung}:}\\[1ex]
  \multicolumn{2}{l}{
      $
      \begin{array}{rllr}
	\max & \displaystyle\sum_{i\in I} p_i\cdot x_i & & \\[3ex]
	s.t. & \displaystyle\sum_{i\in I} v_{ri}\cdot x_i \leq c_r & \quad\forall r\in R & \mathrm{(I)}\\
		& x_i \geq 0 & \quad\forall i\in I & \\
      \end{array}
      $
  }
 \end{tabularx}
\end{frame}

\begin{frame}
 \frametitle{Modell vs. Modellinstanz}
 \begin{block}{Modell}
  Die \alert{allgemeine} Darstellung der Problemstruktur mit \alert{allgemeinen} Indexmengen und Parametern.
 \end{block}
 \begin{block}{Modellinstanz}
  Ein \alert{konkretes} Problem, in welchem den Indexmengen und Parametern \alert{konkrete} Werte zugewiesen wurden.
 \end{block}
\end{frame}
