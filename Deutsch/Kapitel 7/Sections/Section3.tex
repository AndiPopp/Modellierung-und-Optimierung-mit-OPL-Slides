\sectionframe{OPL in Programmabläufen}
\begin{frame}
 \frametitle{Beispiel: Cyclic-Staffing-Problem}
 \begin{block}{Algorithmische Lösung für das Cyclic-Staffing-Problem}\footnotesize
  \begin{enumerate}
   \item $P_1$ sei die LP-Relaxation des Cyclic-Staffing-Problems. Löse $P_1$, wenn Optimallösung ganzzahlig Ende, sonst weiter zu 2.
   \item Füge $P_1$ die Nebenbedingung \[\sum_{s\in S}x_s = \left\lfloor\sum_{s\in S}x^*_s\right\rfloor\] hinzu und erhalte $P_2$. Hat $P_2$ eine Lösung Ende, sonst weiter zu 3.
   \item Füge $P_1$ die Nebenbedingung \[\sum_{s\in S}x_s = \left\lceil\sum_{s\in S}x^*_s\right\rceil\] hinzu und erhalte $P_3$. Optimallösung von $P_3$ ist Optimallösung des Cyclic-Staffing-Problems.
  \end{enumerate}
 \end{block}
\end{frame}

\begin{frame}
 \frametitle{Einbindung von OPL/CPLEX in Programmabläufen}
 \footnotesize
 \begin{block}{Schnittstellen}
  \begin{itemize}
    \item Kommandozeilenanwendung, insbesondere \texttt{oplrun}
    \item ILOG Concert-API
    \item CPLEX Callable Library
    \item CPLEX-Spezialschnittstellen
    \item ILOG Script
  \end{itemize}
 \end{block}
 \pause
 \begin{block}{Auswahl von Vorgängen für Anwendungsbeispiele}
  \begin{itemize}
   \item automatisierte Erstellung und Lösung von Modellinstanzen
   \item Auslesen von Entscheidungsvariablen nach der Lösung
   \item Einspielen automatisch generierter Daten in eine Modellinstanz
  \end{itemize}
 \end{block}
\end{frame}

\begin{frame}
 \frametitle{Vorbereitungen im Beispiel des Cyclic-Staffing-Problems}
 Folgende Nebenbedingung werden hinzugefügt:
 \begin{gather*}
  \sum_{s\in S}x_s \leq ub\label{eq:CyclicStaffingUB}\\
  \sum_{s\in S}x_s \geq lb\label{eq:CyclicStaffingLB}
 \end{gather*}
 
 Haben beiden Schrankenparameter den gleichen Wert so erhält man effektiv:
 \[
  \sum_{s\in S}x_s = ub = lb
 \]
\end{frame}

\subsection{Kommandozeile und Datenquellen}
\begin{frame}
 \frametitle{Vor- und Nachteile der Kommandozeilenanwendung}
 \footnotesize
 \begin{block}{Vorteile}
  \begin{itemize}
   \item mit allen Programmiersprachen kombinierbar, die Kommandozeilenanwendungen ausführen können
   \item sehr vielseitig
   \item gut geeignet für akademische Forschung und kommerzielle Prototypen
  \end{itemize}
 \end{block}
 \pause
 \begin{block}{Nachteile}
  \begin{itemize}
   \item müssen entweder mit gegebenen Schnittstellen für Datenquellen kombiniert werden oder Datenein- und Ausgabe muss selbst programmiert werden
   \item empfindlich gegenüber Fehler in der Konfiguration der Plattform
   \item schlecht geeignet für Produktivsysteme
  \end{itemize}
 \end{block}
\end{frame}

\begin{frame}[fragile]
 \frametitle{Beispiel: VBA}
 \begin{block}{Quelltextausschnitt: Befehlsaufruf}
  \begin{lstlisting}[language={[Visual]Basic},showstringspaces=false,numbers=none,basictype=\scriptsize\ttfamily]
'Löse P1
Shell ("oplrun" & modPath & " " & datPath)
  \end{lstlisting}
 \end{block}
 \begin{block}{Quelltextausschnitt: Workaround für leeren Lösungsraum}
    \begin{lstlisting}[language={[Visual]Basic},showstringspaces=false]
'Ergebnis ist nicht ganzzahlig, 
'bereite Schritt 2 vor
dataSheet.Range("Obj").Value = "n"
  \end{lstlisting}
 \end{block}
\end{frame}


\subsection{Die Concert-API am Beispiel Java}
\begin{frame}
 \frametitle{Klassen-Pakete}
 \begin{description}
  \item[ilog.concert] Klassen, die Schnittstellen zu ILOG Concert bereitstellen
  \item[ilog.cplex] Klassen, die Schnittstellen zu ILOG CPLEX bereitstellen
  \item[ilog.cp] Klassen, die Schnittstellen zu ILOG CP Optimizer bereitstellen
  \item[ilog.opl] Klassen, die Schnittstellen zu OPL bereitstellen
 \end{description}
\end{frame}

\begin{frame}
 \frametitle{Wichtige Objekte}
 \begin{description}
  \item[IloOplFactory] die zentrale Klasse zur Konstruktion anderer OPL-Objekte
  \item[IloOplModelSource] eine Modelldatei im Dateisystem
  \item[IloOplErrorHandler] Pipeline zur Ausgabe von OPL-Fehlern und -Warnungen (Standard: System.out)
  \item[IloOplModelDefintion] interne Repräsentation eines Modells
  \item[IloCplex] repräsentiert eine Instanz des CPLEX-Solvers
  \item[IloOplModel] eine Probleminstanz
  \item[IloOplDataSource] Datenquelle für eine Modellinstanz
 \end{description}
\end{frame}

\begin{frame}[fragile]
 \frametitle{Modelle instanzieren und lösen}
 \begin{block}{Beispiel: Cyclic-Staffing-Problem}
\begin{lstlisting}[language=java,numbers=none,basicstyle=\tiny\ttfamily]
IloOplFactory oplF = new IloOplFactory();

IloOplModelSource modelSource =
  oplF.createOplModelSource("CyclicStaffingProblem.mod");

IloOplErrorHandler err = oplF.createOplErrorHandler();
IloOplModelDefinition def = oplF.createOplModelDefinition(
  modelSource, oplF.createOplSettings(err));

IloCplex cplex = oplF.createCplex();

IloOplModel opl = oplF.createOplModel(def, cplex);

IloOplDataSource dataSource =
 oplF.createOplDataSource("CyclicStaffingProblem.dat");
opl.addDataSource(dataSource);

opl.generate();

opl.getCplex().solve();

opl.printSolution(System.out);
\end{lstlisting}
 \end{block}
\end{frame}

\begin{frame}
 \frametitle{Zugriff auf Modellelemente}
 \begin{itemize}
  \item Zugriff auf mittels Klasse \structure{\texttt{IloOplElement}} und Methode \structure{\texttt{getElement(\textsf{String s})}}.
  \item Datentypenübersetzung mittels \structure{\texttt{as}}-Methoden
 \end{itemize}
 \begin{table}
 \centering\scriptsize
 \begin{tabular}{*{3}{>{\ttfamily}l}}
  \toprule
  \sffamily\bfseries OPL-Datentyp & \sffamily\bfseries Concert-Datentyp-Keyword & \sffamily\bfseries Java-Datentyp\\
  \midrule
  %\multicolumn{3}{c}{\cellcolor{gray!10}\footnotesize Primititve Datentypen}\\
  int & {\texttt{Int}} & int\\
  float & {\texttt{Num}} & double\\
  string & {\texttt{Symbol}} & String\\
  \midrule
  %\multicolumn{3}{c}{\cellcolor{gray!10}\footnotesize Abgeleitete Datentypen}\\
  \textrm{Set} & \texttt{Set} & \\
  \textrm{Array} & {\texttt{Map}} & \\
  tuple & {\texttt{Tuple}} & \\
  range & \texttt{Range} & \\
  dvar & {\texttt{Var}} & \\
  dexpr & {\texttt{Expr}} & \\
  \bottomrule
 \end{tabular}
\end{table}
\end{frame}

\begin{frame}[fragile]
 \frametitle{Eigene Datenquellen definieren}
 \begin{itemize}
  \item abstrakte Klasse \structure{\texttt{IloCustomOplDataSource}}
  \item Datenübergabe durch Funktion \structure{\texttt{customRead}}
 \end{itemize}
 \begin{block}{Beispiel: Cyclic-Staffing-Problem}
\begin{lstlisting}[language=java,numbers=none,basicstyle=\tiny\ttfamily]
@Override
public void customRead() {
  //Datahandler-Objekt initialisieren
  IloOplDataHandler handler = getDataHandler();

  //ub übergeben
  handler.startElement("ub");
  handler.addIntItem(this.ub);
  handler.endElement();

  //lb übergeben
  handler.startElement("lb");
  handler.addIntItem(this.lb);
  handler.endElement();
}
\end{lstlisting}
 \end{block}
\end{frame}

\subsection{ILOG Script}
\begin{frame}
 \frametitle{ILOG Script}
 \begin{itemize}
  \item ILOG Script ist eine Erweiterung von JavaScript
  \item Basiert auf Concert
  \item Scripte werden direkt in die Modelldatei geschrieben
  \item \structure{execute}-Blöcke können im Pre- und Postprocessing verwendet werden.
  \begin{itemize}
   \item vereinfachte Syntax
   \item OPL-Variablen können wie Scriptvariablen verwendet werden
  \end{itemize}
  \item Der \structure{main}-Block dient der Ablaufsteuerung. Hier können mehrere Instanzen konstruiert und gelöst werden.
 \end{itemize}
\end{frame}
